\documentclass{article}
\usepackage{graphicx}
\usepackage{amsmath}
\usepackage{amssymb}
\usepackage{amsthm}
\usepackage[margin = 1.5in]{geometry}
\usepackage{multicol}
\usepackage{float}
\usepackage{verbatim}
\newtheorem{theorem}{Theorem}[section]
\newtheorem{corollary}{Corollary}[theorem]
\newtheorem{lemma}[theorem]{Lemma}
\newtheorem{definition}{Definition}[section]
\renewcommand\qedsymbol{$\blacksquare$}

\title{The Black-Scholes Model}
\author{Kevin Phan}
\date{February 2024}

\begin{document}

\maketitle

\begin{abstract}
    The Black-Scholes model and its many variants are a fundamental PDE in continuous-time finance. 
\end{abstract}

\section{Introduction: Terminology \& Basic Concepts}

The Black Scholes was a model developed by two economists in 1973 that began a revolution in Finance.
Another individual, Merton had also developed the model using the Feynman-Kac. 
 
\begin{definition}[Option]
What is an option? I have no idea man.
\end{definition}

I must take some time to define some finance topics in order to proceed without confusing anyone.



\section{Outlining the Black Scholes Model}
We begin by outlining some basic concepts about stochastic differential equations and Brownian motion. 

\section{Applied Example Using Python}

Here, I will use the formula that we have derived to do a brief numerical example using Python.


\end{document}