\documentclass{article}
\usepackage{graphicx}
\usepackage{amsmath}
\usepackage{amssymb}
\usepackage{amsthm}
\usepackage[margin = 1.5in]{geometry}
\usepackage{multicol}
\usepackage{float}
\usepackage{verbatim}
\newtheorem{theorem}{Theorem}[section]
\newtheorem{corollary}{Corollary}[theorem]
\newtheorem{lemma}[theorem]{Lemma}
\newtheorem{definition}{Definition}[section]
\renewcommand\qedsymbol{$\blacksquare$}

\title{The Black-Scholes Model}
\author{Kevin Phan}
\date{February 2024}

\begin{document}


\maketitle

\begin{abstract}
    The Black-Scholes model and its many variants are a fundamental PDE in 
    continuous-time finance. As a result, there has been a special focus to the 
    model in a pedagogical and research perspective. While the Black-Scholes model is not widely used anymore, 
    it serves as an introduction to the power of mathematics in quantative finance. 
\end{abstract}

\section{Introduction: Terminology \& Basic Concepts}

Mathematical Finance is a developed applied field in mathematics that attracts many people from the management sciences, social sciences (Economics), physics and mathematics.

In the world of finance, there are many markets that people can engage in trade in. Stock markets are the most known, 
but there are also bond markets (usually from the government or institutions like universities), currency markets (foreign exchange), and much more. 

The market that will be the focus in this paper are \emph{futures} and \emph{options} markets. As markets have become more complicated, people have 
opted into selling contracts whose value are based off of other assets (such as stocks). These types of contracts are consequently called \emph{financial derivatives}
due to how they base their value off of other assets. 


 
\begin{definition}[European Call Option]
The most simple option where there is a expiration date, 
where the holder may purchase/sell the underlying asset for some pre-arranged exercise price/strike price.
\end{definition}
In particular, there are two types of options that we will cover. One is the 
\subsection*{The Importance of Options}

So why trade in options when you could trade in stocks, bonds, or something else? 



\section{Outlining the Black Scholes Model}
The Black Scholes was a model developed by two economists in 1973 that began a revolution in Finance.
Another individual, Merton had also developed the model using the Feynman-Kac. 

We begin by outlining some basic concepts about stochastic differential equations and Brownian motion. 

\subsection*{Assumptions}
Many concepts are contested in finance, the next concept is one of the most contested.
\begin{definition}[Efficient Market Hypothesis]
    Markets are efficient enough to the point where no excess profit. There is no opportunity for arbitrage.
\end{definition}

\begin{itemize}
\item There are no opportunities for arbitrage
\item There are \emph{risk-free} investments that had a guaranteed returns with no chance of default.\footnote[1]{One may think of government bonds and treasury bills as something that accomplishes this.}
\end{itemize}


\subsection*{Notation}

We will have $V(S,t)$ represent the value of the option (generally). When we discuss call options, we will use $C(S,t)$, and for puts, $P(S,t)$.
$S$ is the underlying value of the asset (which in this case is usually a stock), and $t$ denotes time.

For a European call option, the Black Scholes Equation is:

\begin{equation}
    \frac{\partial V}{\partial t} + \frac{1}{2}\sigma^2S^2 \frac{\partial^2 V}{\partial S^2} + rS\frac{\partial V}{\partial S} = 0 
\end{equation}

% https://www.math.unl.edu/~sdunbar1/TestingMathJax/Lessons/BlackScholes/Solution/solution.html
% this one actually uses separation of variables


% I'm skipping a bunch of things to start writing here, but it will be good

The change of variables contains $V(S,t) = \tilde{V}(\tau, x)$
\begin{align}
    \tau &= \frac{\sigma^2(T-t)}{2} \\
    x &= \log{(\frac{S}{K})}
\end{align}

We do a change of variables and will compute the derivatives accordingly

\begin{align*}
    \frac{\partial V}{\partial t} &= K\frac{\partial \tilde{V}}{\partial \tau}
    \cdot \frac{d\tilde{V}}{dt} \\
    &= K\frac{\partial \tilde{V}}{\partial \tau} \cdot \frac{-\sigma^2}{2}
\end{align*}

Of course, there is a second derivative, but I will skip the work the 
\begin{align*}
    \frac{\partial^2 V}{\partial S^2} &= \frac{\partial }{\partial S} \frac{\partial V}{\partial S} \\
    &= K \frac{\partial \tilde{V}}{\partial x} \cdot \frac{-1}{s^2} + K \frac{\partial^2 \tilde{V}}{\partial x^2} \cdot \frac{1}{S^2}
\end{align*}

The terminal condition is $V(S,t) = \max{(S-K, 0) = \max{(Ke^x - K, 0)}}$
Additionally, 
$V(S,T) = K\tilde{V}(x,0)$ so $\tilde{V}(x,0) = \max{(e^x -1)}$

When we substitute the everything from the change of variables (and do a lot of substititions), we will get:

\begin{equation}
    \frac{\partial \tilde{V}}{\partial \tau} = \frac{\partial^2 \tilde{V}}{\partial x^2} + (k-1)\frac{\partial \tilde{V}}{\partial x} - k\tilde{V}
\end{equation}

We can now work towards a solution for the heat equation. However, we have to make a change of the dependent variable by
\begin{equation}
    \tilde{V} = e^{\alpha x + \beta \tau} u(x,\tau)
\end{equation}

Using the product rule, we can get a new set of $\tilde{V}$ an their partial derivatives.
\begin{align}
    \tilde{V}_{\tau} = \beta e^{\alpha x + \beta \tau} u + e^{\alpha x + \beta \tau} u_{\tau} \\
    \tilde{V}_{x} = \alpha e^{\alpha x + \beta \tau} u + e^{\alpha x + \beta \tau} u_{x} \\
\end{align}
\begin{equation}
    \tilde{V}_{xx} = \alpha^2 e^{\alpha x + \beta \tau} u + 2\alpha e^{\alpha x + \beta \tau} u_{x} + e^{\alpha x + \beta \tau} u_{xx}. 
\end{equation}

Now we have to put these into the differential equation and then simplify. 
S



\begin{equation}
    u(x, \tau) = 
\end{equation}

\subsubsection*{Heat-Equation Solution Representation Formula}




\subsubsection*{Solution and the Black Scholes Formula}


And finally, we have approached the solution of the Black-Scholes equation, arriving at a formula. 
\begin{equation}
    V(S,t) = S\Phi \frac{\log{(S/K)} + (r + \sigma^2/2)(T-t)}{\sigma \sqrt{T-t}}
    - Ke^{r(T-t)} \Phi \frac{\log{(\frac{S}{K})} + (r-\sigma^{2}/2)(T-t)}{\sigma \sqrt{T-t}}
\end{equation}

Though this solution is extremely long and usually isn't seen in popular depictions of the formula. 



% \section{Applied Example Using Python}

% Here, I will use the formula that we have derived to do a brief numerical example using Python.


\section*{Conclusion}

The Black-Scholes equation (and model) is a beautiful example of how mathematics can revolutionize a field. 
However, the Black-Scholes model is not what it is made out to be by many people. To this day, many assumptions involving the model 
(which I have elected to not cover here due to its specificity) are still unproven. The model is also considerably outdated, with other models being developed
that implement stochastic calculus. 

All in all. 


\end{document}