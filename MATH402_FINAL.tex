\documentclass{article}
\usepackage{graphicx}
\usepackage{amsmath}
\usepackage{amssymb}
\usepackage{amsthm}
\usepackage[margin = 1.5in]{geometry}
\usepackage{multicol}
\usepackage{float}
\usepackage{verbatim}
\newtheorem{theorem}{Theorem}[section]
\newtheorem{corollary}{Corollary}[theorem]
\newtheorem{lemma}[theorem]{Lemma}
\newtheorem{definition}{Definition}[section]
\renewcommand\qedsymbol{$\blacksquare$}

\title{The Black-Scholes Model}
\author{Kevin Phan}
\date{February 2024}

\begin{document}


\maketitle

\begin{abstract}
    The Black-Scholes model and its many variants are a fundamental PDE in 
    continuous-time finance. I outline the fundamental concepts that lie behind the Black-Scholes and options trading, 
    then solve the Black-Scholes using a change of variables approach. 
    After, I implement an applied example of the Black-Scholes formula using Python and discuss the implications of the model. 
\end{abstract}

\section{Introduction: Terminology \& Basic Concepts}

Mathematical Finance is a developed applied field in mathematics that attracts many people from the management sciences, social sciences (Economics), physics and mathematics.

In the world of finance, there are many markets that people can engage in trade in. Stock markets are the most known, 
but there are also bond markets (usually from the government or institutions like universities), currency markets (foreign exchange), and much more. 

The market that will be the focus in this paper are \emph{futures} and \emph{options} markets. As markets have become more complicated, people have 
opted into selling contracts whose value are based off of other assets (such as stocks). These types of contracts are consequently called \emph{financial derivatives}
due to how they base their value off of other assets. 


 
\begin{definition}[European Call Option]
The most simple option where there is a expiration date, 
where the holder may purchase/sell the underlying asset for some pre-arranged exercise price/strike price.
\end{definition}
In particular, there are two types of options that we will cover. One is the call option and the other is the put option. The call option is when the contract involves an asset to be bought,
and a put option deals with the right to sell the asset.

\subsubsection*{Call Options}
I'll give a small example to show how call options work. 
Suppose we have the choice to purchase a call option that has these conditions outlined.
\begin{itemize}
\item Date: May 1, 2024 
\item Holder may purchase one share of AAPL (Apple)
\item Strike Price: \$150USD
\item Price of the Option: \$20USD
\end{itemize}

Suppose on May 1st, Apple will release a new iPhone with infinite battery life. 
The stock of Apple would skyrocket to \$250, and investors that had previously invested in Apple would enjoy a large gain.
However, holding Apple stock can be disadvantageous (say, when the stock depreciates in price), 
and so if you bought an option in April anticipating the growth (perhaps through insider trading or dumb luck)
then you can cash in on the option.

If you bought the option, and then exercised the option (meaning you bought the underlying stock, AAPL), 
then you would pay a total of \$170 USD (option price + strike price) to buy the option and the stock.

If you immediately sell the AAPL stock on May 1st, you would gain a profit of $\$250-\$170 = \$80$. 

\subsubsection*{Put Options}
Put options can be interpreted as the opposite of call options. Let's continue with the previous example of the AAPL stock,
but now we are in July 1st. After having a profitable campaign selling the new iPhone, EU has sued AAPL for anticompetitive behavior, 
driving down the stock price of AAPL from \$250 to \$170. 

Thankfully, we were able to get our hands onto a put option to make the best out of this situation.
\begin{itemize}
\item Date: July 1st, 2024
\item Holder may sell one share of AAPL (Apple)
\item Strike Price: \$220 USD
\item Price of Option: \$30 USD
\end{itemize}

Once we exercise this put option, then we would spend a total of \$30 
to obtain the option. If we exercise the put option and buy the stock (which is now \$170 after the EU lawsuit),
then we would have a total spend of \$200. However, we can quickly flip the asset in order to make a profit, by selling it for the prescribed \$220. 
Thus, we make a \$20 in profit from a situation that would be a loss for most investors.


\subsection*{The Importance of Options}
Options are extremely useful for investors due to their ability to hedge and speculate. 

Hedging in when investor covers their losses using other assets that increase in value. A commonly cited example is the mutual fund, which is composed
of the stock of hundreds to thousands of companies in order to deal with the potential losses in a specific company or industry. The same can be done with options.
For the case of a call option, if the stock does not appreciate in price (let's say that the new iPhone was garbage), then the investor could just not exercise the option and take a loss of \$30.
While this is bad, it is much better than buying the stock for \$170 and then seeing the stock value drop. 

The speculative potential of options is relatively obvious. The investor is betting money on whether the asset value will increase (for call options) or will drop (put options).
If an investor had purchased a call option, then they would hope that the price increases over time. For put options, the investor is hoping that the asset will depreciate in value.
Thus, there is a elevated risk when it comes to options compared to other assets.




\section{Outlining the Black Scholes Model}
The Black Scholes was a model developed by two economists in 1973 that began a revolution in Finance.
Another individual, Merton had also developed the model using the Feynman-Kac. 

We begin by outlining some basic concepts about stochastic differential equations and Brownian motion. 

\subsection*{Assumptions}
Many concepts are contested in finance, but the next concept is one of the most contested.
\begin{definition}[Efficient Market Hypothesis]
    Markets are efficient enough to the point where no excess profit. There is no opportunity for arbitrage. Meaning they cannot simply buy and resell some random asset for a profit.
\end{definition}
Additionally, there are more specific assumptions that the model makes.
\begin{itemize}
\item There are no opportunities for arbitrage
\item There are \emph{risk-free} investments that had a guaranteed returns with no chance of default.\footnote[1]{One may think of government bonds and treasury bills as something that accomplishes this.}
\item The asset price follows a lognormal random walk.\footnote[2]{Hints that the underlying theory is stochastic calculus.}
\item The risk-free interest rate and the asset volatility are known over the entire time of the option.
\end{itemize}

In simple words, arbitrage is when an individual buys a good at a cheap price and sells it for a higher price, making profit while doing very little work. A well-known example of this is drop-shipping, 
which is the act of buying goods from manufacturers or other sites for a low price (Alibaba, Temu, etc.), then reselling them on platforms that usually charge a higher price (Amazon, Etsy, etc.).

Interest rates don't necessarily exist in stocks, bonds, and other asset classes. We traditionally think of interest when dealing with savings accounts, loans, and mortgages. However, certain 
assets (such as government bonds) have a fixed rate of return, which acts similarly to an interest rate we may see in our personal lives.

\subsection*{Notation}



We will have $V(S,t)$ represent the value of the option (generally). When we discuss call options, we will use $C(S,t)$, and for puts, $P(S,t)$.
$S$ is the underlying value of the asset (which in this case is usually a stock), and $t$ denotes time.
\begin{itemize}
    \item $\sigma$, volatility of the underlying asset
    \item $E$, the exercise/strike price
    \item $T$, the expiry date (when you can exercise the option)
    \item $r$, the risk-free interest rate
    \end{itemize}

For a European call option, the Black Scholes Equation is:

\begin{equation}
    \frac{\partial V}{\partial t} + \frac{1}{2}\sigma^2S^2 \frac{\partial^2 V}{\partial S^2} + rS\frac{\partial V}{\partial S} = 0 
\end{equation}

% https://www.math.unl.edu/~sdunbar1/TestingMathJax/Lessons/BlackScholes/Solution/solution.html
% this one actually uses separation of variables/change of variables.

We also have to consider the initial data where $V(0,t) = 0$ and $V(S,t) ~ S$ as $S \to \infty$
Finally, when $V(S,T) = \max{(K-S,0)}$


\section*{Solution of the Black-Scholes Equation}

This is probably for the appendix, but I'll do it here for now.
\subsubsection*{Heat-Equation Solution Representation Formula}
% I'm pretty sure it is the solution to one of these initial value problems
% if we have 

We'll have to take a step back towards the regular heat equation and consider how the boundary-conditions can 
change the solution. 

Suppose,
\begin{equation}
    u_{\tau} = u_{xx}, \text{ where }u(x,0) = u_{0}(x)
\end{equation}
Where $x \in \mathbb{R} $ and $\tau \ge 0$

If we impose further requirements on the system, such as when 
\begin{itemize}
\item $u_0(x)$ only has a finite number of jump discontinuities
\item $lim_{|x| \to \infty} u_{0}(x)e^{-ax^2} = 0$  for any $a > 0$ 
\item $lim_{|x| \to \infty} u(x, \tau)e^{-ax^2} = 0$  for any $a > 0$ 
\end{itemize} 
While these conditions are not necessary to achieve the equation that we will get, it is important in general for the rest of mathematical finance.


If we use the dirac delta function, then we can obtain the fundamental solution of the heat equation. 

\begin{align*}
    \Phi(x,\tau) = \frac{1}{\sqrt{4\pi \tau}} exp(\frac{-x^2}{4\tau})
\end{align*}

Then to get the general solution of the heat equation, can insert $\Phi(x, \tau)$ into:
\begin{align*}
    u(x, \tau) = \frac{1}{2\sqrt{\pi \tau}} \int_{-\infty}^{\infty} exp(\frac{-(x-s)^2}{4\tau}) ds
\end{align*}

We will use this later when we have to solve after applying the change of dependent variable.


% NOTHING IS SUPPOSED TO BE HERE...



We do a change of variables and will compute the derivatives accordingly


% I'm skipping a bunch of things to start writing here, but it will be good

The change of variables contains $V(S,t) = \tilde{V}(\tau, x)$
\begin{align}
    \tau &= \frac{\sigma^2(T-t)}{2} \\
    x &= \log{(\frac{S}{K})}
\end{align}


\begin{align*}
    \frac{\partial V}{\partial t} &= K\frac{\partial \tilde{V}}{\partial \tau}
    \cdot \frac{d\tilde{V}}{dt} \\
    &= K\frac{\partial \tilde{V}}{\partial \tau} \cdot \frac{-\sigma^2}{2}
\end{align*}

Of course, there is a second derivative, but I will skip the work the 
\begin{align*}
    \frac{\partial^2 V}{\partial S^2} &= \frac{\partial }{\partial S} \frac{\partial V}{\partial S} \\
    &= K \frac{\partial \tilde{V}}{\partial x} \cdot \frac{-1}{s^2} + K \frac{\partial^2 \tilde{V}}{\partial x^2} \cdot \frac{1}{S^2}
\end{align*}

The terminal condition is $V(S,t) = \max{(S-K, 0) = \max{(Ke^x - K, 0)}}$
Additionally, 
$V(S,T) = K\tilde{V}(x,0)$ so $\tilde{V}(x,0) = \max{(e^x -1)}$

When we substitute the everything from the change of variables (and do a lot of substititions), we will get:

\begin{equation}
    \frac{\partial \tilde{V}}{\partial \tau} = \frac{\partial^2 \tilde{V}}{\partial x^2} + (k-1)\frac{\partial \tilde{V}}{\partial x} - k\tilde{V}
\end{equation}

We can now work towards a solution for the heat equation. However, we have to make a change of the dependent variable by
\begin{equation}
    \tilde{V}(\tau, x) = e^{\alpha x + \beta \tau} u(x,\tau)
\end{equation}

With this, we can get a new set of $\tilde{V}$ and their partial derivatives.
\begin{align}
    \tilde{V}_{\tau} = \beta e^{\alpha x + \beta \tau} u + e^{\alpha x + \beta \tau} u_{\tau} \\
    \tilde{V}_{x} = \alpha e^{\alpha x + \beta \tau} u + e^{\alpha x + \beta \tau} u_{x} \\
\end{align}
\begin{equation}
    \tilde{V}_{xx} = \alpha^2 e^{\alpha x + \beta \tau} u + 2\alpha e^{\alpha x + \beta \tau} u_{x} + e^{\alpha x + \beta \tau} u_{xx}. 
\end{equation}

Now we have to put these into the differential equation and then simplify. The $e^{\alpha x + \beta \tau}$ should cancel out in this. 

\begin{equation}
    \beta V + V_{\tau} = \alpha^2 V + 2\alpha V_{x} + 
\end{equation}



\begin{equation}
    u(x, \tau) = 
\end{equation}

\subsubsection*{Final Solution and the Black Scholes Formula}


And finally, we have approached the solution of the Black-Scholes equation, arriving at a formula. 
\begin{equation}
    V(S,t) = S\Phi \frac{\log{(S/K)} + (r + \sigma^2/2)(T-t)}{\sigma \sqrt{T-t}}
    - Ke^{r(T-t)} \Phi \frac{\log{(\frac{S}{K})} + (r-\sigma^{2}/2)(T-t)}{\sigma \sqrt{T-t}}
\end{equation}

Though this solution is extremely long and usually isn't seen in popular depictions of the formula. 
We can define two new variables $d_1$ and $d_2$ to be new parameters.



\section{Applied Example Using Python}
We do this to show how the call or put work with the thing and more motivation behind why the thing is important.
% Here, I will use the formula that we have derived to do a brief numerical example using Python.


\section*{Conclusion}

The Black-Scholes equation (and model) is a beautiful example of how mathematics can revolutionize a field. 
However, the Black-Scholes model is not what it is made out to be by many people. To this day, many assumptions involving the model 
(which I have elected to not cover here due to its specificity) are still unproven. The model is also considerably outdated, with other models being developed
that implement stochastic calculus. 

All in all. 


\end{document}